\documentclass[12pt,letterpaper]{article}
\usepackage{graphicx,textcomp}
\usepackage{natbib}
\usepackage{setspace}
\usepackage{fullpage}
\usepackage{color}
\usepackage[reqno]{amsmath}
\usepackage{amsthm}
\usepackage{fancyvrb}
\usepackage{amssymb,enumerate}
\usepackage[all]{xy}
\usepackage{endnotes}
\usepackage{lscape}
\newtheorem{com}{Comment}
\usepackage{float}
\usepackage{hyperref}
\newtheorem{lem} {Lemma}
\newtheorem{prop}{Proposition}
\newtheorem{thm}{Theorem}
\newtheorem{defn}{Definition}
\newtheorem{cor}{Corollary}
\newtheorem{obs}{Observation}
\usepackage[compact]{titlesec}
\usepackage{dcolumn}
\usepackage{tikz}
\usetikzlibrary{arrows}
\usepackage{multirow}
\usepackage{xcolor}
\newcolumntype{.}{D{.}{.}{-1}}
\newcolumntype{d}[1]{D{.}{.}{#1}}
\definecolor{light-gray}{gray}{0.65}
\usepackage{url}
\usepackage{listings}
\usepackage{color}

% from lshort.pdf
\usepackage{verbatim}
% includes comment env
\definecolor{codegreen}{rgb}{0,0.6,0}
\definecolor{codegray}{rgb}{0.5,0.5,0.5}
\definecolor{codepurple}{rgb}{0.58,0,0.82}
\definecolor{backcolour}{rgb}{0.95,0.95,0.92}

\lstdefinestyle{mystyle}{
	backgroundcolor=\color{backcolour},   
	commentstyle=\color{codegreen},
	keywordstyle=\color{magenta},
	numberstyle=\tiny\color{codegray},
	stringstyle=\color{codepurple},
	basicstyle=\footnotesize,
	breakatwhitespace=false,         
	breaklines=true,                 
	captionpos=b,                    
	keepspaces=true,                 
	numbers=left,                    
	numbersep=5pt,                  
	showspaces=false,                
	showstringspaces=false,
	showtabs=false,                  
	tabsize=2
}
\lstset{style=mystyle}
\newcommand{\Sref}[1]{Section~\ref{#1}}
\newtheorem{hyp}{Hypothesis}

\title{Problem Set 2}
\date{Due: October 16, 2022}
\author{Applied Stats/Quant Methods 1}

\begin{document}
	\maketitle
	
	\begin{comment}
	\section*{Instructions}
\begin{itemize}
	\item Please show your work! You may lose points by simply writing in the answer. If the problem requires you to execute commands in \texttt{R}, please include the code you used to get your answers. Please also include the \texttt{.R} file that contains your code. If you are not sure if work needs to be shown for a particular problem, please ask.
	\item Your homework should be submitted electronically on GitHub.
	\item This problem set is due before 23:59 on Sunday October 16, 2022. No late assignments will be accepted.
	\item Total available points for this homework is 80.
\end{itemize}

	\end{comment}
	
	\vspace{.5cm}
	\section*{Question 1 (40 points): Political Science}
		\vspace{.25cm}
	The following table was created using the data from a study run in a major Latin American city.\footnote{Fried, Lagunes, and Venkataramani (2010). ``Corruption and Inequality at the Crossroad: A Multimethod Study of Bribery and Discrimination in Latin America. \textit{Latin American Research Review}. 45 (1): 76-97.} As part of the experimental treatment in the study, one employee of the research team was chosen to make illegal left turns across traffic to draw the attention of the police officers on shift. Two employee drivers were upper class, two were lower class drivers, and the identity of the driver was randomly assigned per encounter. The researchers were interested in whether officers were more or less likely to solicit a bribe from drivers depending on their class (officers use phrases like, ``We can solve this the easy way'' to draw a bribe). The table below shows the resulting data.

\newpage
\begin{table}[h!]
	\centering
	\begin{tabular}{l | c c c }
		& Not Stopped & Bribe requested & Stopped/given warning \\
		\\[-1.8ex] 
		\hline \\[-1.8ex]
		Upper class & 14 & 6 & 7 \\
		Lower class & 7 & 7 & 1 \\
		\hline
	\end{tabular}
\end{table}

\begin{enumerate}

	\item [(a)]
	The $\chi^2$ test statistic is calculated as follows:\\
	%\vspace{7cm}
	
	Read in the data as a matrix.
	\lstinputlisting[language=R, firstline=61, lastline=61]{PS02.R}  

	Calculate the expected values, then calculate the difference between the observed 
	and expected values for each sub-category.  Calculate the contribution to the 
	$\chi^2$ statistic. 
	(expected = number in class * number of outcomes / total number;
		difference = observed - expected; contribution = difference$^2$/expected) %\caret
		
		For example, for the sub-category `Upper Class' and `Not Stopped':
	
	\begin{table}[htb]
		\centering
		\begin{tabular}{l | l}
			\multicolumn{2}{c}{Upper Class, Not Stopped} \\
			\\[-1.8ex] 
			\hline \\[-1.8ex]
			observed  &  14  \\
			expected  &  13.5 = (27 * 21 / 42)\\
			difference  &  0.5  = (14 - 13.5)\\
			chi sq contribution  &  0.0185  = (0.5 )$^2$ / 13.5\\
			
		\end{tabular}
	\end{table}
	
	\lstinputlisting[language=R, firstline=77, lastline=108]{PS02.R}  

	\item [(b)]
	Now calculate the p-value from the test statistic you just created (in \texttt{R}).\footnote{Remember frequency should be $>$ 5 for all cells, but let's calculate the p-value here anyway.}  What do you conclude if $\alpha = 0.1$?\\
	
	% todo check cell minimum for chi-sq - expected only or all??
	
	\lstinputlisting[language=R, firstline=125, lastline=126]{PS02.R}  
	
	\begin{verbatim}
	The p-value is 15.02%, alpha is 10%
	We cannot reject the null hypothesis that the two sets are from the
same population
	1 observed cell(s) with less than 5 values
	\end{verbatim}

  The observed and expected values are shown in Figure~\ref{fig:obs_exp}	
	\begin{figure}
		  \includegraphics{obs_exp.png}
		  \caption{Observed vs Expected values for traffic stop. ns = Not Stopped; sgw = Stopped Given Warning; br = Bribe Requested}\label{fig:obs_exp}
	\end{figure}

  The results of the builtin \texttt{R} chisq.test function are as follows:
  \begin{verbatim}
  
	Pearson's Chi-squared test

data:  observed
X-squared = 3.7912, df = 2, p-value = 0.1502
\end{verbatim}
	\newpage
	\item [(c)] The standardized residuals are set out in the table below:
	\vspace{1cm}

	
% Table created by stargazer v.5.2.3 by Marek Hlavac, Social Policy Institute. E-mail: marek.hlavac at gmail.com
% Date and time: Mon, Oct 10, 2022 - 20:14:46
\begin{table}[h] \centering 
  \caption{Standardised Residuals} 
  \label{StandardisedResiduals} 
\begin{tabular}{@{\extracolsep{5pt}} cccc} 
\\[-1.8ex]\hline \\[-1.8ex] 
 & NotStopped & BribeRequested & StoppedGivenWarning \\ 
\hline \\[-1.8ex] 
UpperClass & $0.32$ & $$-$1.64$ & $1.52$ \\ 
LowerClass & $$-$0.32$ & $1.64$ & $$-$1.52$ \\ 
\hline \\[-1.8ex] 
\end{tabular} 
\end{table}  

	
	%\vspace{7cm}
	\item [(d)] How might the standardized residuals help you interpret the results?  

    The biggest contribution to the residuals was from the `Bribe Requested' variable - 
    fewer upper class individuals were expected to hand over bribes.  The difference 
    between the two groups appears to be a combination of fewer upper class drivers
    being expected to hand over bribes and more of them being given a warning instead 
    the opposite outcome occuring for lower class drivers.
    
  
\end{enumerate}

\newpage

\section*{Question 2 (40 points): Economics}
Chattopadhyay and Duflo were interested in whether women promote different policies than men.\footnote{Chattopadhyay and Duflo. (2004). ``Women as Policy Makers: Evidence from a Randomized Policy Experiment in India. \textit{Econometrica}. 72 (5), 1409-1443.} Answering this question with observational data is pretty difficult due to potential confounding problems (e.g. the districts that choose female politicians are likely to systematically differ in other aspects too). Hence, they exploit a randomized policy experiment in India, where since the mid-1990s, $\frac{1}{3}$ of village council heads have been randomly reserved for women. A subset of the data from West Bengal can be found at the following link: \url{https://raw.githubusercontent.com/kosukeimai/qss/master/PREDICTION/women.csv}\\

\noindent Each observation in the data set represents a village and there are two villages associated with one GP (i.e. a level of government is called "GP"). Figure~\ref{fig:women_desc}
below shows the names and descriptions of the variables in the dataset. The authors hypothesize that female politicians are more likely to support policies female voters want. Researchers found that more women complain about the quality of drinking water than men. You need to estimate the effect of the reservation policy on the number of new or repaired drinking water facilities in the villages.
\vspace{.5cm}
\begin{figure}[h!]
	\caption{\footnotesize{Names and description of variables from Chattopadhyay and Duflo (2004).}}
	\vspace{.5cm}
	\centering
	\label{fig:women_desc}
	\includegraphics[width=1.0\textwidth]{graphics/women_desc.png}
\end{figure}		

\newpage
\begin{enumerate}
	\item [(a)] State a null and alternative (two-tailed) hypothesis. 
	
	\begin{description}
	  \item [Null] The reservation policy has no effect on the number of new or 
	  repaired drinking water facilities in the villages.
	  \item [Alternate] The reservation policy does have an effect on the number of new or 
	  repaired drinking water facilities in the villages.
	\end{description}
	
	\begin{comment}
	
	The data is as follows: 

| [,3] `reserved`~~ binary variable indicating whether the GP was reserved for women leaders or not
| [,4] `female` ~~binary variable indicating whether the GP had a female leader or not
| [,5] `irrigation` ~~variable measuring the number of new or repaired irrigation facilities in the village since the reserve policy started
| [,6] `water` ~~variable measuring the number of new or repaired drinking-water facilities in the village since the reserve policy started

  \end{comment}
	
	%\vspace{6cm}
	\item [(b)] Bivariate regression to test this hypothesis:.

  Import the data.
	\lstinputlisting[language=R, firstline=215, lastline=215]{PS02.R}  
	
	Use the builtin \texttt{R} function \texttt{lm} to investigate the relationship
	between the number of new or repaired drinking water facilities in the villages
	and the binary variable indicating whether the GP was reserved for women leaders or not.
	\lstinputlisting[language=R, firstline=222, lastline=222]{PS02.R}  
	
	
	This results in the following output:
	
	
% Table created by stargazer v.5.2.3 by Marek Hlavac, Social Policy Institute. E-mail: marek.hlavac at gmail.com
% Date and time: Tue, Oct 11, 2022 - 20:16:03
\begin{table}[!htbp] \centering 
  \caption{Pearson Linear Regression - Water ~ Reserved} 
  \label{water_reserved} 
\begin{tabular}{@{\extracolsep{5pt}}lc} 
\\[-1.8ex]\hline \\[-1.8ex] 
\\[-1.8ex] & water \\ 
\hline \\[-1.8ex] 
 reserved & 9.252$^{**}$ \\ 
  & (3.948) \\ 
  Constant & 14.738$^{***}$ \\ 
  & (2.286) \\ 
 N & 322 \\ 
R$^{2}$ & 0.017 \\ 
Adjusted R$^{2}$ & 0.014 \\ 
Residual Std. Error & 33.446 (df = 320) \\ 
F Statistic & 5.493$^{**}$ (df = 1; 320) \\ 
\hline \\[-1.8ex] 
\multicolumn{2}{l}{$^{*}$p $<$ .1; $^{**}$p $<$ .05; $^{***}$p $<$ .01} \\ 
\end{tabular} 
\end{table}  

	
	
%	\vspace{6cm}
	\item [(c)] Interpret the coefficient estimate for reservation policy. 
\end{enumerate}
  The 
\begin{figure}[h!]
	\caption{\footnotesize{Boxplot of number of drinking water projects, grouped by reserved}}
	\vspace{.5cm}
	\centering
	\label{fig:water_reserved}
	\includegraphics[width=1.0\textwidth]{water_boxplot.png}
\end{figure}		

\newpage
\appendix{Appendix - Code}
	\lstinputlisting[language=R]{PS02.R}  


\end{document}
